\chapter{Conclusion and Outlook}\label{sec:outlook}
% Nassiri and McCall
In this novel approach, the \ac{cp} data by Bray \textit{et al.} were used to predict certain \acl{p} assays that were selected for their relation to cytotoxicity. Nassiri and McCall\cite{Nassiri2018} used model performance as an indication for a common \ac{moa}. A similar approach is considered in this project. By comparatively analyzing performance metrics from an \ac{rfc} model among different descriptors, the relation of the bioassay to cellular morphology as captured by the \ac{cp} assay is inferred.\\
% results from comparative analysis
Within a comparative analysis of the different modelling approaches, twelve bioassay exhibit elevated predictive potential via \ac{cp} descriptors which were able to outperform \acp{ecfp} on nearly every metric. However, the specificity and sensitivity showed different behaviour in comparison to \ac{auc}, \acl{ba} and \ac{mcc}. The \ac{tpr} scored especially high for within the group of \acl{hpa}. That leads to the conclusion that \ac{cp} descriptors have a higher chance that a positively labelled compound is indeed positive. This characteristic is beneficial for toxicity prediction since the ability to correctly predict toxic (positive) compounds can prevent unnecessary testing and harm. The \ac{tnr} was found to be better predicted using either \acp{ecfp} or the feature engineered descriptor set. However, the \acl{sf} could not achieve an improvement over the individual predictive capabilities. It was also shown that \acp{ecfp} and \ac{cp} descriptors complement each other, which is in agreement with the findings from Lapins and Spjuth\cite{Lapins2019}. For drug safety projects, it is suggested to examine if \ac{cp} descriptors characterize the endpoint well. If so, feature engineering can be omitted for \ac{ml} approaches comparable to the one presented here. Predicting the \ac{tnr} is presumably better achievable via \acp{ecfp}.\\
% channel enrichment
A new enrichment metric developed from feature importance analysis is presented here. It measures the fluorescence channel-wise assay group standard deviation. This enrichment metric connects the results from feature engineering with cellular morphology, and it was found that for four out of five fluorescent channels, enrichment could be measured. This approach can be used as the first entry point when working with \ac{cp} data to categorize bioassays before prediction runs and focus on the assays that are estimated to be well characterized by \ac{cp} data.
% Lapins et spjuth
Lapins and Spjuth\cite{Lapins2019} used \ac{moa} and targets to annotate compounds in their work and used \ac{cp}, structural fingerprints and gene expression profiles as descriptors. A similar approach was chosen in the sense that structural fingerprints and \ac{cp} features were used as descriptors. Furthermore, their use of genetic information as a descriptor led to the idea that phenotypic annotations \ac{go} terms could be used to investigate the biological mechanisms responsible for performance differences within the \acl{p} bioassays. Opposite to Lapins et Spjuth\cite{Lapins2019}, the genetic information was not used as a descriptor but to further annotate the targets supplied by the \acl{p} assays.\\
The phenotypic annotations have been manually generated and connect the bioassays readouts to cellular processes (e.g. signalling, proteome regulation, genotoxicity etc.) that are likely to result in morphological changes. By calculating the phenotypic enrichment for \acl{hpa} and \acl{lpa}, it was found that endpoints that relate to genome integrity, \ac{dna}-repair, genotoxicity, cell-death, apoptosis, cell stress, toxins and immune response are generally better described by \ac{cp} descriptors.\\
Since the results agreed with the findings from the channel enrichment, the phenotypic analysis can be considered a confirmatory argument. However, its generalizability is hindered by its irreproducibility and critically imperfect knowledge. In an attempt to mitigate these shortcomings, the \ac{go} term analysis was conducted with results affirmative of the channel enrichment. Many \ac{go} terms found for the \acl{hpa} are connected to \ac{dna}, \ac{rna} and macromolecule synthesis. The caveat for \ac{go} term analysis stems from the relatively small sample size. Only six bioassays within the \acl{hpa} are probing gene products, therefore qualifying for this analysis. Nonetheless, it is suggested that \ac{go} terms should be included for a mechanistic and cellular understanding of the prediction performance when \ac{cp} data is in use, especially if a large number of endpoints are included in the analysis.