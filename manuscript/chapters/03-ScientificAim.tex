\chapter{Scientific Aim}\label{sec:scientificaim}
% most general notion of the scientific aim
This project aims to generate heuristics that simplify working with \ac{cp} data. Generally, \ac{cp} data can be used as descriptors to predict targets obtained from compound associated biochemical readouts. The mechanistic relation between a biochemical readout and its predictivity via \ac{cp} descriptors is poorly understood. Therefore, this work aims at understanding the results obtained from \ac{rfc} prediction and linking the results to cellular mechanisms and the concept of cytotoxicity.\\
% specify what that means
Conceptually, this means finding bioassays whose endpoints are related to toxicity for annotating the \ac{cp} descriptors. Since this is a comparative approach, structural fingerprints (\acp{ecfp}) are used as another descriptor set. Both annotated data sets are entered into an \ac{ml} model, and this model's discriminatory power is evaluated using generic statistical metrics. Another model is trained using the combined set of descriptors which is evaluated analogously.\\
% specify the approach taken in this work
The detailed procedure starts by preprocessing the \ac{cp} raw image data into a \ac{ml}-ready data frame. Next, the bioassays are selected from \acl{p}\cite{Kim2020}, based on size and their relation to cytotoxicity. Every bioassay data frame is combined with the \ac{cp} data frame to obtain annotated sets containing inputs as well as targets for prediction.\\
For comparison, structural descriptors are generated for all data sets using \texttt{sklearn}-funtionalities\cite{Pedregosa2012}. To this end, the annotations in all data sets are highly imbalanced, comprising mostly samples labelled as inactive. Herein, this problem is tackled by applying \ac{smote} to the data sets in combination with undersampling of the majority label (inactive).\\
Two \ac{rfc} models are trained on each data set, and their predictive power is evaluated. Furthermore, to examine if one descriptor's shortcomings can be mitigated by the other, selected features from both descriptors are combined, and another model is trained and evaluated. Since there is no apparent way to select features manually, statistical methods can be used to conduct features selection meaningfully. \ac{pca}, \ac{mrmr} and random forest feature importance are applied to score and select features from each set of descriptors for merging. Eventually, a fourth \ac{rfc} model is trained and evaluated using the complete feature sets from both descriptors.\\
Based on the prediction evaluation and feature selection process, a rigorous analysis is conducted to explain the results with respect to cellular morphology and cytotoxicity. The focus lies in detecting and analyzing prediction similarities and dissimilarities between either descriptor sets, bioassays or groups characterized by their performance. Thereby, patterns can be detected that transform into heuristics which facilitate the application of \ac{cp} data to \ac{ml} problems in the future.