\chapter{Summary}\label{sec:summary}
% introduction to toxicity
The pharmaceutical industry is centred around small molecules and their effects. Apart from the curative effect, the absence of adverse or toxicological effects is cardinal. However, toxicity is at least as elusive as it is important. A simple definition is: 'toxicology is the science of adverse effects of chemicals on living organisms'.\cite{Singh2018} However, this definition comprises several caveats. What is the organism? Where do therapeutic and adverse effects start and end? Even for the simplest organisms' toxicity, cytotoxicity, the mechanisms are manifold and difficult to unravel. Hence, it remains obscure which characteristics a compound has to combine to be labelled as toxic.\\
% introduction to cell painting
One attempt to illuminate these characteristics are novel \ac{cp} assays. For a \ac{cp} assay, cells are perturbed by libraries of small compounds, which might affect the cellular morphology before images are taken via automated fluorescence microscopy. Five fluorescent channels are used for imaging, and these channels correspond to certain cell organelles\cite{Carpenter2006}. Therefore \ac{cp} data contains information about cell structure variations caused by each compound. Which sub-information is actually valuable within these morphological fingerprints remains elusive. Therefore a significant part of the project presented here is dedicated to exploring the \ac{cp} data and their predictive capabilities comparatively. They will be compared against different descriptors for a variety of bioassays. The \ac{cp} data used in this project contains roughly \num{30000} compounds and \num{1800} features.\cite{Bray2017}\\
% introduce structural fingerprints
In chemistry, the structure determines the properties of a compound or substance. Therefore, apart from \ac{cp}, structural fingerprints are used as a benchmark descriptor set for comparison. In this project \acp{ecfp} were used to encode the compounds' structures as numerical features.\\
% introduction to pubchem assays
This work is concerned with morphological changes that correspond to toxicity. Thus, the \ac{cp} data were combined with toxicological endpoints from specific assays selected from the \acl{p} database. The selection process implemented a minimum number of active compounds, a size criterion and the occurrence of toxicologically relevant targets.\cite{Mervin2016}\\
% the procedure
After the selected assays were combined with each of their descriptors, machine learning models were trained, and their predictive power was evaluated against specific metrics. The predictions can be divided into four cycles. In the first cycle, the \ac{cp} data are used as descriptors, the second cycle used the structural fingerprints, and the third cycle used a subset of both. A rigorous feature engineering process selected the subsets. The last cycle skipped the feature engineering and combined all \ac{cp} and \ac{ecfp} descriptors into one large set of inputs.\\
% results
% metrics
The evaluation of the prediction metrics illuminates which strengths and shortcomings the morphological fingerprints feature compared to the structural fingerprints. It turned out that there are two groups of assays: those \acl{p} assays that are generally better predicted with \ac{cp} features and those that have higher predictive potential when using \ac{ecfp}. Additionally, it was revealed that \ac{ecfp} comprise higher specificity compared to \ac{cp} data which show higher sensitivity on the other hand. A high sensitivity means the prediction rarely mislabels a sample as negative (e.g. non-toxic) compared to the number of correctly labelled positive samples (e.g. toxic compounds.). Based on these results, \ac{cp} is better suited for toxicity prediction and drug safety evaluations since the mislabelled, positive compound can lead to expenses or even damage to health.\\
% channel enrichment
Furthermore, based on the data from fluorescent channels, an enrichment measure was introduced and calculated for the aforementioned two groups of \acl{p} assays. This enrichment connects predictive performance with cell organelle activity. The hypothesis was that \acl{p} assays, reliably predictable from \ac{cp} data, should exhibit increased enrichment, which was the case for four out of five fluorescence microscopy channels.\\
% phenotypic enrichment
As a next step, phenotypic terms were manually generated to categorize the different \acl{p} assays. These terms corresponded to cellular mechanisms or morphological processes and were generated unbiasedly. Nevertheless, they are subject to human error. The phenotypic annotations that are found to be enriched for successful modelling approaches might guide the pre-selection of bioassays in future projects. The enrichment analysis of phenotypic annotations detected that \acl{p} assays that could be well predicted via \ac{cp} data are related to immune response, genotoxicity and genome regulation and cell death.\\
% Go terms
Finally, the assays are assigned \ac{go} terms obtained from the \ac{go} database.\cite{Ashburner2000,Carbon2020} These terms comprise a controlled, structured vocabulary that explicitly describes the molecular function and biological processes of a given gene product. For \acl{p} assays associated with a protein target, the \ac{go} terms are collected. If an assay is particularly well predicted via \ac{cp} descriptors, the associated \ac{go} terms can relate this finding to cellular function. Even though the analysis with go terms suffers from a minimal sample size, it was found that \ac{cp} related assays usually correspond to processes concerning \ac{dna} and other macromolecules. This finding is in good agreement with the analysis of the channel enrichment as well as the phenotypic enrichment.
%% hier passt richtig gut ne abbildung hin vom generellen ablauf


\section{Zusammenfassung}
% Einführung in Pubchem-Assays
Diese Arbeit befasst sich mit zellul\"aren, morphologischen Veränderungen in Zusammenhang Toxizität. \ac{cp}-Daten wurden hierbei mit toxikologischen Endpunkten aus spezifischen Assays kombiniert, die aus der \acl{p}-Datenbank ausgewählt wurden. Das Auswahlverfahren implementierte eine Mindestanzahl von Wirkstoffen, ein Größenkriterium und das Auftreten toxikologisch relevanter Endpunkte.\cite{Mervin2016}\\
% das Verfahren
Nachdem die ausgewählten Assays mit ihren Deskriptoren kombiniert worden waren, wurden Modelle für \ac{ml} trainiert und ihre Vorhersagekraft anhand spezifischer Kenngr\"o{\ss}en bewertet. Die Vorhersagen können in vier Zyklen unterteilt werden. Im ersten Zyklus wurden die \ac{cp}-Daten als Deskriptoren verwendet, im zweiten Zyklus wurden strukturelle Merkmale verwendet, und im dritten Zyklus wurde eine Teilmenge beider verwendet. Ein ausgiebiger Feature-Engineering-Prozess wählte die Teilmengen aus. Im letzten Zyklus wurde das Feature-Engineering übersprungen und alle \ac{cp}- und \ac{ecfp}-Deskriptoren zu einem großen Datensatz zusammengefasst.\\
% Ergebnisse
% Metriken
Die Auswertung der Vorhersagemetriken zeigt, welche Stärken und Mängel die morphologischen Fingerabdrücke im Vergleich zu den strukturellen Merkmalen aufweisen. Es stellte sich heraus, dass es zwei Gruppen von Assays gibt: jene \acl{p}-Assays, die mit \ac{cp}-Daten im Allgemeinen besser vorhergesagt werden können, und jene, die bei Verwendung von \ac{ecfp} ein höheres Vorhersagepotential haben. Zusätzlich wurde gezeigt, dass \acp{ecfp} eine höhere Spezifität aufweisen als \ac{cp}-Daten, die andererseits eine höhere Empfindlichkeit zeigen. Eine hohe Empfindlichkeit bedeutet f\"ur eine Vorhersage, dass eine Probe im Vergleich zur Anzahl korrekt markierter positiver Proben (z. B. toxische Verbindungen) selten falsch als negativ (z. B. nicht toxisch) vorausgesagt wird. Basierend auf diesen Ergebnissen sind \ac{cp}-Daten besser für die Vorhersage der Toxizität und die Bewertung der Arzneimittelsicherheit geeignet, da eine falsch ausgewiesene positive Verbindung zu Kosten oder sogar zu Gesundheitsschäden führen kann. \\
% Kanalanreicherung
Darüber hinaus wurde basierend auf den Daten der Fluoreszenzmikroskopiekanäle eine Enrichment-Gr\"o{\ss}e eingeführt und für die oben genannten zwei Gruppen von \acl{p}-Assays berechnet. Diese Enrichment-Gr\"o{\ss}e verbindet die Vorhersageleistung mit der Aktivität der Zellorganellen. Die Hypothese war, dass \acl{p}-Assays, die zuverlässig aus \ac{cp}-Daten vorhersagbar sind, eine erhöhte Enrichment-Gr\"o{\ss}e aufweisen sollten, was bei vier von fünf Fluoreszenzmikroskopiekanälen der Fall war.\\
% phänotypische Anreicherung
Als nächster Schritt wurden phänotypische Kennw\"orter manuell generiert, um die verschiedenen \acl{p}-Assays zu kategorisieren. Diese Begriffe entsprachen zellulären Mechanismen oder morphologischen Prozessen und wurden unvoreingenommen generiert. Trotzdem unterliegen sie menschlichen Fehlern. Die phänotypischen Annotationen, die für erfolgreiche \ac{ml} Modelle angereichert sind, könnten die Vorauswahl von Bioassays in zukünftigen Projekten vereinfachen. Die Enrichment-Analyse phänotypischer Annotationen ergab, dass \acl{p}-Assays, die über \ac{cp}-Daten gut vorhergesagt werden konnten, mit Immunantworten, Genotoxizität und Genomregulation sowie Zelltod zusammenhängen.\\
% Go Begriffe
Schließlich werden den Assays \ac{go}-Begriffe zugewiesen, die aus der \ac{go}-Datenbank stammen. \cite{Ashburner2000, Carbon2020} Diese Begriffe umfassen ein kontrolliertes, strukturiertes Vokabular, das die molekulare Funktion und die biologischen Prozesse eines bestimmten Genprodukts explizit beschreibt. Für \acl{p}-Assays, sofern sie einem Protein Target zugeordnet sind, wurden die \ac{go}-Begriffe gesammelt. Wenn ein Assay über \ac{cp}-Deskriptoren besonders gut vorhergesagt wird, können die zugehörigen \ac{go}-Terme diesen Befund mit der Zellfunktion in Beziehung setzen. Obwohl die Analyse mit GO-Begriffen durch eine kleine Stichprobengröße eingeschr\"ankt sind, wurde festgestellt, dass \ac{cp}-bezogene Assays normalerweise Prozessen entsprechen, die \ac{dna} und andere Makromoleküle betreffen. Dieser Befund stimmt gut mit dem Enrichment der Fluoreszenzmikroskopiekanälen sowie den phänotypischen Annotationen überein.
%% hier passt richtig gut ne abbildung hin vom generellen ablauf